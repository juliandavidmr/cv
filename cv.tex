% LaTeX Curriculum Vitae Template
%
% Copyright (C) 2004-2009 Jason Blevins <jrblevin@sdf.lonestar.org>
% http://jblevins.org/projects/cv-template/
%
% You may use use this document as a template to create your own CV
% and you may redistribute the source code freely. No attribution is
% required in any resulting documents. I do ask that you please leave
% this notice and the above URL in the source code if you choose to
% redistribute this file.

\documentclass[letterpaper]{article}

\usepackage{hyperref}
\usepackage{geometry}
\usepackage{url}
\usepackage{hanging}
\usepackage{parskip}

% Comment the following lines to use the default Computer Modern font
% instead of the Palatino font provided by the mathpazo package.
% Remove the 'osf' bit if you don't like the old style figures.
\usepackage[T1]{fontenc}
\usepackage[sc,osf]{mathpazo}

% Set your name here
\def\name{Juli\'an David Mora Ramos}

% Replace this with a link to your CV if you like, or set it empty
% (as in \def\footerlink{}) to remove the link in the footer:
\def\footerlink{http://jblevins.org/projects/cv-template/}

% The following metadata will show up in the PDF properties
\hypersetup{
  colorlinks = true,
  urlcolor = black,
  pdfauthor = {\name},
  pdfkeywords = {programming, code, mathematics},
  pdftitle = {\name: CV: Juli\'an David Mora Ramos},
  pdfsubject = {Curriculum Vitae},
  pdfpagemode = UseNone
}

\geometry{
  body={6.5in, 8.5in},
  left=1.0in,
  top=1.25in
}

% Customize page headers
\pagestyle{myheadings}
\markright{\name}
\thispagestyle{empty}

% Custom section fonts
\usepackage{sectsty}
\sectionfont{\rmfamily\mdseries\Large}
\subsectionfont{\rmfamily\mdseries\itshape\large}

% Other possible font commands include:
% \ttfamily for teletype,
% \sffamily for sans serif,
% \bfseries for bold,
% \scshape for small caps,
% \normalsize, \large, \Large, \LARGE sizes.

% Don't indent paragraphs.
\setlength\parindent{0em}

% Make lists without bullets
\renewenvironment{itemize}{
  \begin{list}{}{
    \setlength{\leftmargin}{1.5em}
  }
}{
  \end{list}
}

\begin{document}

% Place name at left
{\huge \name}

% Alternatively, print name centered and bold:
%\centerline{\huge \bf \name}

\vspace{0.25in}

\begin{minipage}{0.45\linewidth}
	\href{http://www.udla.edu.co}{Universidad de la Amazonia} \\
	Florencia, Caquet\'a. 180001\\
	Instituci\'on educativa San Lorenzo \\
	Suaza, Huila. 416080 \\
 
\end{minipage}
\begin{minipage}{0.45\linewidth}
  \begin{tabular}{ll}
    Email: & \href{mailto:anlijudavid@hotmail.com}{\tt anlijudavid@hotmail.com} \\
    P\'agina: & \href{https://juliandavidmr.github.io}{\tt https://juliandavidmr.github.io} \\
  \end{tabular}
\end{minipage}

\section*{Intereses de investigaci\'on}
\begin{itemize}
\item Me interesan las nuevas tecnolog\'ias, software, temas de actualidad, educaci\'on, open source. Me motiva aprender de todo un poco, y me apasiona adquirir nuevos conocimientos para ayudar a los dem\'as.
\end{itemize} 

\section*{Educaci\'on}

\begin{itemize}
  \item Ing. Ingenier\'ia de sistemas. Facultad de Ingenier\'ia. Universidad de la Amazonia, 2012 -- actualidad.
  
	  \leftskip 0.5in
	  \parindent -0.5in
	  \subitem Tesis: \href{https://github.com/juliandavidmr/Grado}{\emph{Integraci\'on de servicios virtuales: una alternativa para la gesti\'on de ambientes de aprendizaje en la modalidad de Educaci\'on a distancia de la Universidad de la Amazonia}}. \textit{(En proceso)} 2017 -- actualidad.

  \leftskip 0in
  \item Tec. T\'ecnico en mantenimiento y reparaci\'on de equipos de audio y video. T\'ecnica en Electr\'onica. Servicio Nacional de Aprendizaje SENA, 2010 -- 2011.
  
  \item Diplomados/Certificados:
  
  	  \leftskip 0.5in
	  \subitem Ergonom\'ia en los ambientes de trabajo. SENA. 2010.
	  \subitem T\'ecnicas de dise\~no y construcci\'on de circuitos electr\'onicos impresos PCB. SENA. 2011.
	  \subitem Estructura del lenguaje de programaci\'on C++ (Nivel I). SENA. 2012.
	  \subitem Variables y estructuras de control en la programaci\'on orientada a objetos: JAVA. SENA. 2012.
  
\end{itemize}

\section*{Investigaciones}
\begin{itemize}
	\item {\bf Proyectos actuales}
	
	\leftskip 0.5in
	\parindent -0.5in
	\subitem Toc app. Aplicaci\'on m\'ovil h\'ibrida para la comunidad estudiantil y administrativa del Departamento de Educaci\'on a Distancia (2017 -- actualidad)
	\subitem Sistema de transferencia de datos entre el sistema misional Chair\'a (Universidad de la Amazonia) y el LMS Moodle (Departamento de Educaci\'on a Distancia) (2017 -- 2018)
	
	
	\leftskip 0in
	\item {\bf Otros proyectos}
	
	\leftskip 0.5in
	\parindent -0.5in
	\subitem RP. Parser de lenguaje de programaci\'on. Abstracci\'on del lenguaje de programaci\'on php. Sintaxis limpia, libre de elementos repetitivos de PHP, incluido el soporte para muchas funciones php y segmentos que lo ayudan a crear archivos con menos contenido de c\'odigo.(2007 -- actualidad)
	\\ \\
	\subitem Open source: 20 paquetes de NodeJS publicadas en npm en los que se destacan:

	\leftskip 1in
	\parindent -0.5in
	\subitem sails-inverse-model: Generador de archivos para creacion de CRUD en SailsJS. \textit{(6,097 descargas)}
	\subitem vue-frame: Creaci\'on de interfaces con iframes en VueJS \textit{(3,191 descargas)}
	\subitem ionis: Herramienta para ayudar en la certificación de aplicaciones m\'oviles h\'ibridas de Ionic Framework v2. \textit{(522 descargas)}
	\subitem sylver: paquete liviano para operaciones matem\'aticas complejas \textit{(293 descargas)}
\end{itemize}


\section*{Skills}
\begin{itemize}
\item {\bf Programaci\'on}
\subitem Javascript, JAVA, C\#, PHP, Python, C++, PLSQL, Shell, TypeScript, Tex. Conocimientos previos en Kotlin, Dart, Golang y Ruby.

\item {\bf Herramientas y librer\'ias}
\subitem Frontend: Angular \textit{(Todas las versiones, tambi\'en Ionic Framework)}, VueJS, ReactJS \textit{(Incluyendo contenedores de control de estado, como Redux)}.
\subitem Backend: NodeJS \textit{(SailsJS, MeteorJS, ExpressJS, NestJS, KoaJS)}, ASP.NET (C\#),  bases de datos relacionales \textit{(MySQL, PostgreSQL, Oracle)}, bases de datos no relacionales \textit{(MongoDB, Neo4j)}

\item {\bf Methods}

\leftskip 0.5in
\parindent -0.5in

\subitem MIT Classes: Calculus (I\&II), Differential Equations, Quantitative Research Methods (I\&II), Introduction to Computer Science (Python) 
\subitem{Harvard Classes: Topics in Quantitative Methods (Gov 2002), Advanced Quantitative Research Methodology (Gov 2001)}

\end{itemize}

\bigskip

% Footer
\begin{center}
  \begin{footnotesize}
    Last updated: \today \\
  \end{footnotesize}
\end{center}

\end{document}
